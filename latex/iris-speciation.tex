\documentclass[11pt]{article}

    \usepackage[breakable]{tcolorbox}
    \usepackage{parskip} % Stop auto-indenting (to mimic markdown behaviour)
    
    \usepackage{iftex}
    \ifPDFTeX
    	\usepackage[T1]{fontenc}
    	\usepackage{mathpazo}
    \else
    	\usepackage{fontspec}
    \fi

    % Basic figure setup, for now with no caption control since it's done
    % automatically by Pandoc (which extracts ![](path) syntax from Markdown).
    \usepackage{graphicx}
    % Maintain compatibility with old templates. Remove in nbconvert 6.0
    \let\Oldincludegraphics\includegraphics
    % Ensure that by default, figures have no caption (until we provide a
    % proper Figure object with a Caption API and a way to capture that
    % in the conversion process - todo).
    \usepackage{caption}
    \DeclareCaptionFormat{nocaption}{}
    \captionsetup{format=nocaption,aboveskip=0pt,belowskip=0pt}

    \usepackage{float}
    \floatplacement{figure}{H} % forces figures to be placed at the correct location
    \usepackage{xcolor} % Allow colors to be defined
    \usepackage{enumerate} % Needed for markdown enumerations to work
    \usepackage{geometry} % Used to adjust the document margins
    \usepackage{amsmath} % Equations
    \usepackage{amssymb} % Equations
    \usepackage{textcomp} % defines textquotesingle
    % Hack from http://tex.stackexchange.com/a/47451/13684:
    \AtBeginDocument{%
        \def\PYZsq{\textquotesingle}% Upright quotes in Pygmentized code
    }
    \usepackage{upquote} % Upright quotes for verbatim code
    \usepackage{eurosym} % defines \euro
    \usepackage[mathletters]{ucs} % Extended unicode (utf-8) support
    \usepackage{fancyvrb} % verbatim replacement that allows latex
    \usepackage{grffile} % extends the file name processing of package graphics 
                         % to support a larger range
    \makeatletter % fix for old versions of grffile with XeLaTeX
    \@ifpackagelater{grffile}{2019/11/01}
    {
      % Do nothing on new versions
    }
    {
      \def\Gread@@xetex#1{%
        \IfFileExists{"\Gin@base".bb}%
        {\Gread@eps{\Gin@base.bb}}%
        {\Gread@@xetex@aux#1}%
      }
    }
    \makeatother
    \usepackage[Export]{adjustbox} % Used to constrain images to a maximum size
    \adjustboxset{max size={0.9\linewidth}{0.9\paperheight}}

    % The hyperref package gives us a pdf with properly built
    % internal navigation ('pdf bookmarks' for the table of contents,
    % internal cross-reference links, web links for URLs, etc.)
    \usepackage{hyperref}
    % The default LaTeX title has an obnoxious amount of whitespace. By default,
    % titling removes some of it. It also provides customization options.
    \usepackage{titling}
    \usepackage{longtable} % longtable support required by pandoc >1.10
    \usepackage{booktabs}  % table support for pandoc > 1.12.2
    \usepackage[inline]{enumitem} % IRkernel/repr support (it uses the enumerate* environment)
    \usepackage[normalem]{ulem} % ulem is needed to support strikethroughs (\sout)
                                % normalem makes italics be italics, not underlines
    \usepackage{mathrsfs}
    

    
    % Colors for the hyperref package
    \definecolor{urlcolor}{rgb}{0,.145,.698}
    \definecolor{linkcolor}{rgb}{.71,0.21,0.01}
    \definecolor{citecolor}{rgb}{.12,.54,.11}

    % ANSI colors
    \definecolor{ansi-black}{HTML}{3E424D}
    \definecolor{ansi-black-intense}{HTML}{282C36}
    \definecolor{ansi-red}{HTML}{E75C58}
    \definecolor{ansi-red-intense}{HTML}{B22B31}
    \definecolor{ansi-green}{HTML}{00A250}
    \definecolor{ansi-green-intense}{HTML}{007427}
    \definecolor{ansi-yellow}{HTML}{DDB62B}
    \definecolor{ansi-yellow-intense}{HTML}{B27D12}
    \definecolor{ansi-blue}{HTML}{208FFB}
    \definecolor{ansi-blue-intense}{HTML}{0065CA}
    \definecolor{ansi-magenta}{HTML}{D160C4}
    \definecolor{ansi-magenta-intense}{HTML}{A03196}
    \definecolor{ansi-cyan}{HTML}{60C6C8}
    \definecolor{ansi-cyan-intense}{HTML}{258F8F}
    \definecolor{ansi-white}{HTML}{C5C1B4}
    \definecolor{ansi-white-intense}{HTML}{A1A6B2}
    \definecolor{ansi-default-inverse-fg}{HTML}{FFFFFF}
    \definecolor{ansi-default-inverse-bg}{HTML}{000000}

    % common color for the border for error outputs.
    \definecolor{outerrorbackground}{HTML}{FFDFDF}

    % commands and environments needed by pandoc snippets
    % extracted from the output of `pandoc -s`
    \providecommand{\tightlist}{%
      \setlength{\itemsep}{0pt}\setlength{\parskip}{0pt}}
    \DefineVerbatimEnvironment{Highlighting}{Verbatim}{commandchars=\\\{\}}
    % Add ',fontsize=\small' for more characters per line
    \newenvironment{Shaded}{}{}
    \newcommand{\KeywordTok}[1]{\textcolor[rgb]{0.00,0.44,0.13}{\textbf{{#1}}}}
    \newcommand{\DataTypeTok}[1]{\textcolor[rgb]{0.56,0.13,0.00}{{#1}}}
    \newcommand{\DecValTok}[1]{\textcolor[rgb]{0.25,0.63,0.44}{{#1}}}
    \newcommand{\BaseNTok}[1]{\textcolor[rgb]{0.25,0.63,0.44}{{#1}}}
    \newcommand{\FloatTok}[1]{\textcolor[rgb]{0.25,0.63,0.44}{{#1}}}
    \newcommand{\CharTok}[1]{\textcolor[rgb]{0.25,0.44,0.63}{{#1}}}
    \newcommand{\StringTok}[1]{\textcolor[rgb]{0.25,0.44,0.63}{{#1}}}
    \newcommand{\CommentTok}[1]{\textcolor[rgb]{0.38,0.63,0.69}{\textit{{#1}}}}
    \newcommand{\OtherTok}[1]{\textcolor[rgb]{0.00,0.44,0.13}{{#1}}}
    \newcommand{\AlertTok}[1]{\textcolor[rgb]{1.00,0.00,0.00}{\textbf{{#1}}}}
    \newcommand{\FunctionTok}[1]{\textcolor[rgb]{0.02,0.16,0.49}{{#1}}}
    \newcommand{\RegionMarkerTok}[1]{{#1}}
    \newcommand{\ErrorTok}[1]{\textcolor[rgb]{1.00,0.00,0.00}{\textbf{{#1}}}}
    \newcommand{\NormalTok}[1]{{#1}}
    
    % Additional commands for more recent versions of Pandoc
    \newcommand{\ConstantTok}[1]{\textcolor[rgb]{0.53,0.00,0.00}{{#1}}}
    \newcommand{\SpecialCharTok}[1]{\textcolor[rgb]{0.25,0.44,0.63}{{#1}}}
    \newcommand{\VerbatimStringTok}[1]{\textcolor[rgb]{0.25,0.44,0.63}{{#1}}}
    \newcommand{\SpecialStringTok}[1]{\textcolor[rgb]{0.73,0.40,0.53}{{#1}}}
    \newcommand{\ImportTok}[1]{{#1}}
    \newcommand{\DocumentationTok}[1]{\textcolor[rgb]{0.73,0.13,0.13}{\textit{{#1}}}}
    \newcommand{\AnnotationTok}[1]{\textcolor[rgb]{0.38,0.63,0.69}{\textbf{\textit{{#1}}}}}
    \newcommand{\CommentVarTok}[1]{\textcolor[rgb]{0.38,0.63,0.69}{\textbf{\textit{{#1}}}}}
    \newcommand{\VariableTok}[1]{\textcolor[rgb]{0.10,0.09,0.49}{{#1}}}
    \newcommand{\ControlFlowTok}[1]{\textcolor[rgb]{0.00,0.44,0.13}{\textbf{{#1}}}}
    \newcommand{\OperatorTok}[1]{\textcolor[rgb]{0.40,0.40,0.40}{{#1}}}
    \newcommand{\BuiltInTok}[1]{{#1}}
    \newcommand{\ExtensionTok}[1]{{#1}}
    \newcommand{\PreprocessorTok}[1]{\textcolor[rgb]{0.74,0.48,0.00}{{#1}}}
    \newcommand{\AttributeTok}[1]{\textcolor[rgb]{0.49,0.56,0.16}{{#1}}}
    \newcommand{\InformationTok}[1]{\textcolor[rgb]{0.38,0.63,0.69}{\textbf{\textit{{#1}}}}}
    \newcommand{\WarningTok}[1]{\textcolor[rgb]{0.38,0.63,0.69}{\textbf{\textit{{#1}}}}}
    
    
    % Define a nice break command that doesn't care if a line doesn't already
    % exist.
    \def\br{\hspace*{\fill} \\* }
    % Math Jax compatibility definitions
    \def\gt{>}
    \def\lt{<}
    \let\Oldtex\TeX
    \let\Oldlatex\LaTeX
    \renewcommand{\TeX}{\textrm{\Oldtex}}
    \renewcommand{\LaTeX}{\textrm{\Oldlatex}}
    % Document parameters
    % Document title
    \title{iris-speciation}
    
    
    
    
    
% Pygments definitions
\makeatletter
\def\PY@reset{\let\PY@it=\relax \let\PY@bf=\relax%
    \let\PY@ul=\relax \let\PY@tc=\relax%
    \let\PY@bc=\relax \let\PY@ff=\relax}
\def\PY@tok#1{\csname PY@tok@#1\endcsname}
\def\PY@toks#1+{\ifx\relax#1\empty\else%
    \PY@tok{#1}\expandafter\PY@toks\fi}
\def\PY@do#1{\PY@bc{\PY@tc{\PY@ul{%
    \PY@it{\PY@bf{\PY@ff{#1}}}}}}}
\def\PY#1#2{\PY@reset\PY@toks#1+\relax+\PY@do{#2}}

\expandafter\def\csname PY@tok@w\endcsname{\def\PY@tc##1{\textcolor[rgb]{0.73,0.73,0.73}{##1}}}
\expandafter\def\csname PY@tok@c\endcsname{\let\PY@it=\textit\def\PY@tc##1{\textcolor[rgb]{0.25,0.50,0.50}{##1}}}
\expandafter\def\csname PY@tok@cp\endcsname{\def\PY@tc##1{\textcolor[rgb]{0.74,0.48,0.00}{##1}}}
\expandafter\def\csname PY@tok@k\endcsname{\let\PY@bf=\textbf\def\PY@tc##1{\textcolor[rgb]{0.00,0.50,0.00}{##1}}}
\expandafter\def\csname PY@tok@kp\endcsname{\def\PY@tc##1{\textcolor[rgb]{0.00,0.50,0.00}{##1}}}
\expandafter\def\csname PY@tok@kt\endcsname{\def\PY@tc##1{\textcolor[rgb]{0.69,0.00,0.25}{##1}}}
\expandafter\def\csname PY@tok@o\endcsname{\def\PY@tc##1{\textcolor[rgb]{0.40,0.40,0.40}{##1}}}
\expandafter\def\csname PY@tok@ow\endcsname{\let\PY@bf=\textbf\def\PY@tc##1{\textcolor[rgb]{0.67,0.13,1.00}{##1}}}
\expandafter\def\csname PY@tok@nb\endcsname{\def\PY@tc##1{\textcolor[rgb]{0.00,0.50,0.00}{##1}}}
\expandafter\def\csname PY@tok@nf\endcsname{\def\PY@tc##1{\textcolor[rgb]{0.00,0.00,1.00}{##1}}}
\expandafter\def\csname PY@tok@nc\endcsname{\let\PY@bf=\textbf\def\PY@tc##1{\textcolor[rgb]{0.00,0.00,1.00}{##1}}}
\expandafter\def\csname PY@tok@nn\endcsname{\let\PY@bf=\textbf\def\PY@tc##1{\textcolor[rgb]{0.00,0.00,1.00}{##1}}}
\expandafter\def\csname PY@tok@ne\endcsname{\let\PY@bf=\textbf\def\PY@tc##1{\textcolor[rgb]{0.82,0.25,0.23}{##1}}}
\expandafter\def\csname PY@tok@nv\endcsname{\def\PY@tc##1{\textcolor[rgb]{0.10,0.09,0.49}{##1}}}
\expandafter\def\csname PY@tok@no\endcsname{\def\PY@tc##1{\textcolor[rgb]{0.53,0.00,0.00}{##1}}}
\expandafter\def\csname PY@tok@nl\endcsname{\def\PY@tc##1{\textcolor[rgb]{0.63,0.63,0.00}{##1}}}
\expandafter\def\csname PY@tok@ni\endcsname{\let\PY@bf=\textbf\def\PY@tc##1{\textcolor[rgb]{0.60,0.60,0.60}{##1}}}
\expandafter\def\csname PY@tok@na\endcsname{\def\PY@tc##1{\textcolor[rgb]{0.49,0.56,0.16}{##1}}}
\expandafter\def\csname PY@tok@nt\endcsname{\let\PY@bf=\textbf\def\PY@tc##1{\textcolor[rgb]{0.00,0.50,0.00}{##1}}}
\expandafter\def\csname PY@tok@nd\endcsname{\def\PY@tc##1{\textcolor[rgb]{0.67,0.13,1.00}{##1}}}
\expandafter\def\csname PY@tok@s\endcsname{\def\PY@tc##1{\textcolor[rgb]{0.73,0.13,0.13}{##1}}}
\expandafter\def\csname PY@tok@sd\endcsname{\let\PY@it=\textit\def\PY@tc##1{\textcolor[rgb]{0.73,0.13,0.13}{##1}}}
\expandafter\def\csname PY@tok@si\endcsname{\let\PY@bf=\textbf\def\PY@tc##1{\textcolor[rgb]{0.73,0.40,0.53}{##1}}}
\expandafter\def\csname PY@tok@se\endcsname{\let\PY@bf=\textbf\def\PY@tc##1{\textcolor[rgb]{0.73,0.40,0.13}{##1}}}
\expandafter\def\csname PY@tok@sr\endcsname{\def\PY@tc##1{\textcolor[rgb]{0.73,0.40,0.53}{##1}}}
\expandafter\def\csname PY@tok@ss\endcsname{\def\PY@tc##1{\textcolor[rgb]{0.10,0.09,0.49}{##1}}}
\expandafter\def\csname PY@tok@sx\endcsname{\def\PY@tc##1{\textcolor[rgb]{0.00,0.50,0.00}{##1}}}
\expandafter\def\csname PY@tok@m\endcsname{\def\PY@tc##1{\textcolor[rgb]{0.40,0.40,0.40}{##1}}}
\expandafter\def\csname PY@tok@gh\endcsname{\let\PY@bf=\textbf\def\PY@tc##1{\textcolor[rgb]{0.00,0.00,0.50}{##1}}}
\expandafter\def\csname PY@tok@gu\endcsname{\let\PY@bf=\textbf\def\PY@tc##1{\textcolor[rgb]{0.50,0.00,0.50}{##1}}}
\expandafter\def\csname PY@tok@gd\endcsname{\def\PY@tc##1{\textcolor[rgb]{0.63,0.00,0.00}{##1}}}
\expandafter\def\csname PY@tok@gi\endcsname{\def\PY@tc##1{\textcolor[rgb]{0.00,0.63,0.00}{##1}}}
\expandafter\def\csname PY@tok@gr\endcsname{\def\PY@tc##1{\textcolor[rgb]{1.00,0.00,0.00}{##1}}}
\expandafter\def\csname PY@tok@ge\endcsname{\let\PY@it=\textit}
\expandafter\def\csname PY@tok@gs\endcsname{\let\PY@bf=\textbf}
\expandafter\def\csname PY@tok@gp\endcsname{\let\PY@bf=\textbf\def\PY@tc##1{\textcolor[rgb]{0.00,0.00,0.50}{##1}}}
\expandafter\def\csname PY@tok@go\endcsname{\def\PY@tc##1{\textcolor[rgb]{0.53,0.53,0.53}{##1}}}
\expandafter\def\csname PY@tok@gt\endcsname{\def\PY@tc##1{\textcolor[rgb]{0.00,0.27,0.87}{##1}}}
\expandafter\def\csname PY@tok@err\endcsname{\def\PY@bc##1{\setlength{\fboxsep}{0pt}\fcolorbox[rgb]{1.00,0.00,0.00}{1,1,1}{\strut ##1}}}
\expandafter\def\csname PY@tok@kc\endcsname{\let\PY@bf=\textbf\def\PY@tc##1{\textcolor[rgb]{0.00,0.50,0.00}{##1}}}
\expandafter\def\csname PY@tok@kd\endcsname{\let\PY@bf=\textbf\def\PY@tc##1{\textcolor[rgb]{0.00,0.50,0.00}{##1}}}
\expandafter\def\csname PY@tok@kn\endcsname{\let\PY@bf=\textbf\def\PY@tc##1{\textcolor[rgb]{0.00,0.50,0.00}{##1}}}
\expandafter\def\csname PY@tok@kr\endcsname{\let\PY@bf=\textbf\def\PY@tc##1{\textcolor[rgb]{0.00,0.50,0.00}{##1}}}
\expandafter\def\csname PY@tok@bp\endcsname{\def\PY@tc##1{\textcolor[rgb]{0.00,0.50,0.00}{##1}}}
\expandafter\def\csname PY@tok@fm\endcsname{\def\PY@tc##1{\textcolor[rgb]{0.00,0.00,1.00}{##1}}}
\expandafter\def\csname PY@tok@vc\endcsname{\def\PY@tc##1{\textcolor[rgb]{0.10,0.09,0.49}{##1}}}
\expandafter\def\csname PY@tok@vg\endcsname{\def\PY@tc##1{\textcolor[rgb]{0.10,0.09,0.49}{##1}}}
\expandafter\def\csname PY@tok@vi\endcsname{\def\PY@tc##1{\textcolor[rgb]{0.10,0.09,0.49}{##1}}}
\expandafter\def\csname PY@tok@vm\endcsname{\def\PY@tc##1{\textcolor[rgb]{0.10,0.09,0.49}{##1}}}
\expandafter\def\csname PY@tok@sa\endcsname{\def\PY@tc##1{\textcolor[rgb]{0.73,0.13,0.13}{##1}}}
\expandafter\def\csname PY@tok@sb\endcsname{\def\PY@tc##1{\textcolor[rgb]{0.73,0.13,0.13}{##1}}}
\expandafter\def\csname PY@tok@sc\endcsname{\def\PY@tc##1{\textcolor[rgb]{0.73,0.13,0.13}{##1}}}
\expandafter\def\csname PY@tok@dl\endcsname{\def\PY@tc##1{\textcolor[rgb]{0.73,0.13,0.13}{##1}}}
\expandafter\def\csname PY@tok@s2\endcsname{\def\PY@tc##1{\textcolor[rgb]{0.73,0.13,0.13}{##1}}}
\expandafter\def\csname PY@tok@sh\endcsname{\def\PY@tc##1{\textcolor[rgb]{0.73,0.13,0.13}{##1}}}
\expandafter\def\csname PY@tok@s1\endcsname{\def\PY@tc##1{\textcolor[rgb]{0.73,0.13,0.13}{##1}}}
\expandafter\def\csname PY@tok@mb\endcsname{\def\PY@tc##1{\textcolor[rgb]{0.40,0.40,0.40}{##1}}}
\expandafter\def\csname PY@tok@mf\endcsname{\def\PY@tc##1{\textcolor[rgb]{0.40,0.40,0.40}{##1}}}
\expandafter\def\csname PY@tok@mh\endcsname{\def\PY@tc##1{\textcolor[rgb]{0.40,0.40,0.40}{##1}}}
\expandafter\def\csname PY@tok@mi\endcsname{\def\PY@tc##1{\textcolor[rgb]{0.40,0.40,0.40}{##1}}}
\expandafter\def\csname PY@tok@il\endcsname{\def\PY@tc##1{\textcolor[rgb]{0.40,0.40,0.40}{##1}}}
\expandafter\def\csname PY@tok@mo\endcsname{\def\PY@tc##1{\textcolor[rgb]{0.40,0.40,0.40}{##1}}}
\expandafter\def\csname PY@tok@ch\endcsname{\let\PY@it=\textit\def\PY@tc##1{\textcolor[rgb]{0.25,0.50,0.50}{##1}}}
\expandafter\def\csname PY@tok@cm\endcsname{\let\PY@it=\textit\def\PY@tc##1{\textcolor[rgb]{0.25,0.50,0.50}{##1}}}
\expandafter\def\csname PY@tok@cpf\endcsname{\let\PY@it=\textit\def\PY@tc##1{\textcolor[rgb]{0.25,0.50,0.50}{##1}}}
\expandafter\def\csname PY@tok@c1\endcsname{\let\PY@it=\textit\def\PY@tc##1{\textcolor[rgb]{0.25,0.50,0.50}{##1}}}
\expandafter\def\csname PY@tok@cs\endcsname{\let\PY@it=\textit\def\PY@tc##1{\textcolor[rgb]{0.25,0.50,0.50}{##1}}}

\def\PYZbs{\char`\\}
\def\PYZus{\char`\_}
\def\PYZob{\char`\{}
\def\PYZcb{\char`\}}
\def\PYZca{\char`\^}
\def\PYZam{\char`\&}
\def\PYZlt{\char`\<}
\def\PYZgt{\char`\>}
\def\PYZsh{\char`\#}
\def\PYZpc{\char`\%}
\def\PYZdl{\char`\$}
\def\PYZhy{\char`\-}
\def\PYZsq{\char`\'}
\def\PYZdq{\char`\"}
\def\PYZti{\char`\~}
% for compatibility with earlier versions
\def\PYZat{@}
\def\PYZlb{[}
\def\PYZrb{]}
\makeatother


    % For linebreaks inside Verbatim environment from package fancyvrb. 
    \makeatletter
        \newbox\Wrappedcontinuationbox 
        \newbox\Wrappedvisiblespacebox 
        \newcommand*\Wrappedvisiblespace {\textcolor{red}{\textvisiblespace}} 
        \newcommand*\Wrappedcontinuationsymbol {\textcolor{red}{\llap{\tiny$\m@th\hookrightarrow$}}} 
        \newcommand*\Wrappedcontinuationindent {3ex } 
        \newcommand*\Wrappedafterbreak {\kern\Wrappedcontinuationindent\copy\Wrappedcontinuationbox} 
        % Take advantage of the already applied Pygments mark-up to insert 
        % potential linebreaks for TeX processing. 
        %        {, <, #, %, $, ' and ": go to next line. 
        %        _, }, ^, &, >, - and ~: stay at end of broken line. 
        % Use of \textquotesingle for straight quote. 
        \newcommand*\Wrappedbreaksatspecials {% 
            \def\PYGZus{\discretionary{\char`\_}{\Wrappedafterbreak}{\char`\_}}% 
            \def\PYGZob{\discretionary{}{\Wrappedafterbreak\char`\{}{\char`\{}}% 
            \def\PYGZcb{\discretionary{\char`\}}{\Wrappedafterbreak}{\char`\}}}% 
            \def\PYGZca{\discretionary{\char`\^}{\Wrappedafterbreak}{\char`\^}}% 
            \def\PYGZam{\discretionary{\char`\&}{\Wrappedafterbreak}{\char`\&}}% 
            \def\PYGZlt{\discretionary{}{\Wrappedafterbreak\char`\<}{\char`\<}}% 
            \def\PYGZgt{\discretionary{\char`\>}{\Wrappedafterbreak}{\char`\>}}% 
            \def\PYGZsh{\discretionary{}{\Wrappedafterbreak\char`\#}{\char`\#}}% 
            \def\PYGZpc{\discretionary{}{\Wrappedafterbreak\char`\%}{\char`\%}}% 
            \def\PYGZdl{\discretionary{}{\Wrappedafterbreak\char`\$}{\char`\$}}% 
            \def\PYGZhy{\discretionary{\char`\-}{\Wrappedafterbreak}{\char`\-}}% 
            \def\PYGZsq{\discretionary{}{\Wrappedafterbreak\textquotesingle}{\textquotesingle}}% 
            \def\PYGZdq{\discretionary{}{\Wrappedafterbreak\char`\"}{\char`\"}}% 
            \def\PYGZti{\discretionary{\char`\~}{\Wrappedafterbreak}{\char`\~}}% 
        } 
        % Some characters . , ; ? ! / are not pygmentized. 
        % This macro makes them "active" and they will insert potential linebreaks 
        \newcommand*\Wrappedbreaksatpunct {% 
            \lccode`\~`\.\lowercase{\def~}{\discretionary{\hbox{\char`\.}}{\Wrappedafterbreak}{\hbox{\char`\.}}}% 
            \lccode`\~`\,\lowercase{\def~}{\discretionary{\hbox{\char`\,}}{\Wrappedafterbreak}{\hbox{\char`\,}}}% 
            \lccode`\~`\;\lowercase{\def~}{\discretionary{\hbox{\char`\;}}{\Wrappedafterbreak}{\hbox{\char`\;}}}% 
            \lccode`\~`\:\lowercase{\def~}{\discretionary{\hbox{\char`\:}}{\Wrappedafterbreak}{\hbox{\char`\:}}}% 
            \lccode`\~`\?\lowercase{\def~}{\discretionary{\hbox{\char`\?}}{\Wrappedafterbreak}{\hbox{\char`\?}}}% 
            \lccode`\~`\!\lowercase{\def~}{\discretionary{\hbox{\char`\!}}{\Wrappedafterbreak}{\hbox{\char`\!}}}% 
            \lccode`\~`\/\lowercase{\def~}{\discretionary{\hbox{\char`\/}}{\Wrappedafterbreak}{\hbox{\char`\/}}}% 
            \catcode`\.\active
            \catcode`\,\active 
            \catcode`\;\active
            \catcode`\:\active
            \catcode`\?\active
            \catcode`\!\active
            \catcode`\/\active 
            \lccode`\~`\~ 	
        }
    \makeatother

    \let\OriginalVerbatim=\Verbatim
    \makeatletter
    \renewcommand{\Verbatim}[1][1]{%
        %\parskip\z@skip
        \sbox\Wrappedcontinuationbox {\Wrappedcontinuationsymbol}%
        \sbox\Wrappedvisiblespacebox {\FV@SetupFont\Wrappedvisiblespace}%
        \def\FancyVerbFormatLine ##1{\hsize\linewidth
            \vtop{\raggedright\hyphenpenalty\z@\exhyphenpenalty\z@
                \doublehyphendemerits\z@\finalhyphendemerits\z@
                \strut ##1\strut}%
        }%
        % If the linebreak is at a space, the latter will be displayed as visible
        % space at end of first line, and a continuation symbol starts next line.
        % Stretch/shrink are however usually zero for typewriter font.
        \def\FV@Space {%
            \nobreak\hskip\z@ plus\fontdimen3\font minus\fontdimen4\font
            \discretionary{\copy\Wrappedvisiblespacebox}{\Wrappedafterbreak}
            {\kern\fontdimen2\font}%
        }%
        
        % Allow breaks at special characters using \PYG... macros.
        \Wrappedbreaksatspecials
        % Breaks at punctuation characters . , ; ? ! and / need catcode=\active 	
        \OriginalVerbatim[#1,codes*=\Wrappedbreaksatpunct]%
    }
    \makeatother

    % Exact colors from NB
    \definecolor{incolor}{HTML}{303F9F}
    \definecolor{outcolor}{HTML}{D84315}
    \definecolor{cellborder}{HTML}{CFCFCF}
    \definecolor{cellbackground}{HTML}{F7F7F7}
    
    % prompt
    \makeatletter
    \newcommand{\boxspacing}{\kern\kvtcb@left@rule\kern\kvtcb@boxsep}
    \makeatother
    \newcommand{\prompt}[4]{
        {\ttfamily\llap{{\color{#2}[#3]:\hspace{3pt}#4}}\vspace{-\baselineskip}}
    }
    

    
    % Prevent overflowing lines due to hard-to-break entities
    \sloppy 
    % Setup hyperref package
    \hypersetup{
      breaklinks=true,  % so long urls are correctly broken across lines
      colorlinks=true,
      urlcolor=urlcolor,
      linkcolor=linkcolor,
      citecolor=citecolor,
      }
    % Slightly bigger margins than the latex defaults
    
    \geometry{verbose,tmargin=1in,bmargin=1in,lmargin=1in,rmargin=1in}
    
    

\begin{document}
    
    \maketitle
    
    

    
    \hypertarget{final-assignment-analytical-design}{%
\section{Final Assignment: Analytical
Design}\label{final-assignment-analytical-design}}

\hypertarget{task-1}{%
\subsection{Task 1}\label{task-1}}

\hypertarget{a-retrieve-and-clean-data}{%
\subsubsection{a) Retrieve and clean
data}\label{a-retrieve-and-clean-data}}

    \begin{tcolorbox}[breakable, size=fbox, boxrule=1pt, pad at break*=1mm,colback=cellbackground, colframe=cellborder]
\prompt{In}{incolor}{1}{\boxspacing}
\begin{Verbatim}[commandchars=\\\{\}]
\PY{c+c1}{\PYZsh{} import pandas library}
\PY{k+kn}{import} \PY{n+nn}{pandas} \PY{k}{as} \PY{n+nn}{pd}

\PY{c+c1}{\PYZsh{} download dataset}
\PY{n}{dataset} \PY{o}{=} \PY{n}{pd}\PY{o}{.}\PY{n}{read\PYZus{}csv}\PY{p}{(}
    \PY{l+s+s2}{\PYZdq{}}\PY{l+s+s2}{https://archive.ics.uci.edu/ml/machine\PYZhy{}learning\PYZhy{}databases/iris/iris.data}\PY{l+s+s2}{\PYZdq{}}\PY{p}{,}
    \PY{n}{names} \PY{o}{=}\PY{p}{[}\PY{l+s+s2}{\PYZdq{}}\PY{l+s+s2}{sepal\PYZus{}length}\PY{l+s+s2}{\PYZdq{}}\PY{p}{,}\PY{l+s+s2}{\PYZdq{}}\PY{l+s+s2}{sepal\PYZus{}width}\PY{l+s+s2}{\PYZdq{}}\PY{p}{,}\PY{l+s+s2}{\PYZdq{}}\PY{l+s+s2}{petal\PYZus{}length}\PY{l+s+s2}{\PYZdq{}}\PY{p}{,}\PY{l+s+s2}{\PYZdq{}}\PY{l+s+s2}{petal\PYZus{}width}\PY{l+s+s2}{\PYZdq{}}\PY{p}{,} \PY{l+s+s2}{\PYZdq{}}\PY{l+s+s2}{species}\PY{l+s+s2}{\PYZdq{}}\PY{p}{]}
    \PY{p}{)}

\PY{l+s+sd}{\PYZsq{}\PYZsq{}\PYZsq{}}
\PY{l+s+sd}{\PYZsh{} remove attributes we aren\PYZsq{}t using in this exercise}
\PY{l+s+sd}{dataset.drop(\PYZdq{}species\PYZdq{}, inplace=True, axis=1)}
\PY{l+s+sd}{\PYZsq{}\PYZsq{}\PYZsq{}}
\PY{c+c1}{\PYZsh{} remove any rows with missing data}
\PY{n}{dataset}\PY{o}{.}\PY{n}{dropna}\PY{p}{(}\PY{n}{inplace}\PY{o}{=}\PY{k+kc}{True}\PY{p}{)}

\PY{c+c1}{\PYZsh{}show first 10 rows using head() method}
\PY{n}{dataset}\PY{o}{.}\PY{n}{head}\PY{p}{(}\PY{l+m+mi}{10}\PY{p}{)}
\end{Verbatim}
\end{tcolorbox}

            \begin{tcolorbox}[breakable, size=fbox, boxrule=.5pt, pad at break*=1mm, opacityfill=0]
\prompt{Out}{outcolor}{1}{\boxspacing}
\begin{Verbatim}[commandchars=\\\{\}]
   sepal\_length  sepal\_width  petal\_length  petal\_width      species
0           5.1          3.5           1.4          0.2  Iris-setosa
1           4.9          3.0           1.4          0.2  Iris-setosa
2           4.7          3.2           1.3          0.2  Iris-setosa
3           4.6          3.1           1.5          0.2  Iris-setosa
4           5.0          3.6           1.4          0.2  Iris-setosa
5           5.4          3.9           1.7          0.4  Iris-setosa
6           4.6          3.4           1.4          0.3  Iris-setosa
7           5.0          3.4           1.5          0.2  Iris-setosa
8           4.4          2.9           1.4          0.2  Iris-setosa
9           4.9          3.1           1.5          0.1  Iris-setosa
\end{Verbatim}
\end{tcolorbox}
        
    \hypertarget{bextract-statistics}{%
\subsubsection{b)Extract statistics}\label{bextract-statistics}}

\hypertarget{istatistics-of-the-whole-dataset}{%
\paragraph{i)Statistics of the whole
dataset}\label{istatistics-of-the-whole-dataset}}

    \begin{tcolorbox}[breakable, size=fbox, boxrule=1pt, pad at break*=1mm,colback=cellbackground, colframe=cellborder]
\prompt{In}{incolor}{2}{\boxspacing}
\begin{Verbatim}[commandchars=\\\{\}]
\PY{n}{dataset}\PY{o}{.}\PY{n}{describe}\PY{p}{(}\PY{p}{)}
\end{Verbatim}
\end{tcolorbox}

            \begin{tcolorbox}[breakable, size=fbox, boxrule=.5pt, pad at break*=1mm, opacityfill=0]
\prompt{Out}{outcolor}{2}{\boxspacing}
\begin{Verbatim}[commandchars=\\\{\}]
       sepal\_length  sepal\_width  petal\_length  petal\_width
count    150.000000   150.000000    150.000000   150.000000
mean       5.843333     3.054000      3.758667     1.198667
std        0.828066     0.433594      1.764420     0.763161
min        4.300000     2.000000      1.000000     0.100000
25\%        5.100000     2.800000      1.600000     0.300000
50\%        5.800000     3.000000      4.350000     1.300000
75\%        6.400000     3.300000      5.100000     1.800000
max        7.900000     4.400000      6.900000     2.500000
\end{Verbatim}
\end{tcolorbox}
        
    \hypertarget{ii-statistics-for-each-species}{%
\paragraph{ii) Statistics for each
species}\label{ii-statistics-for-each-species}}

    \begin{tcolorbox}[breakable, size=fbox, boxrule=1pt, pad at break*=1mm,colback=cellbackground, colframe=cellborder]
\prompt{In}{incolor}{3}{\boxspacing}
\begin{Verbatim}[commandchars=\\\{\}]
\PY{c+c1}{\PYZsh{} extract statistics for setosa}
\PY{n}{setosa} \PY{o}{=} \PY{n}{dataset}\PY{p}{[}\PY{n}{dataset}\PY{p}{[}\PY{l+s+s2}{\PYZdq{}}\PY{l+s+s2}{species}\PY{l+s+s2}{\PYZdq{}}\PY{p}{]} \PY{o}{==} \PY{l+s+s2}{\PYZdq{}}\PY{l+s+s2}{Iris\PYZhy{}setosa}\PY{l+s+s2}{\PYZdq{}}\PY{p}{]}
\PY{n}{setosa}\PY{o}{.}\PY{n}{describe}\PY{p}{(}\PY{p}{)}
\end{Verbatim}
\end{tcolorbox}

            \begin{tcolorbox}[breakable, size=fbox, boxrule=.5pt, pad at break*=1mm, opacityfill=0]
\prompt{Out}{outcolor}{3}{\boxspacing}
\begin{Verbatim}[commandchars=\\\{\}]
       sepal\_length  sepal\_width  petal\_length  petal\_width
count      50.00000    50.000000     50.000000     50.00000
mean        5.00600     3.418000      1.464000      0.24400
std         0.35249     0.381024      0.173511      0.10721
min         4.30000     2.300000      1.000000      0.10000
25\%         4.80000     3.125000      1.400000      0.20000
50\%         5.00000     3.400000      1.500000      0.20000
75\%         5.20000     3.675000      1.575000      0.30000
max         5.80000     4.400000      1.900000      0.60000
\end{Verbatim}
\end{tcolorbox}
        
    \begin{tcolorbox}[breakable, size=fbox, boxrule=1pt, pad at break*=1mm,colback=cellbackground, colframe=cellborder]
\prompt{In}{incolor}{4}{\boxspacing}
\begin{Verbatim}[commandchars=\\\{\}]
\PY{c+c1}{\PYZsh{} extract statistics for versicolor}
\PY{n}{versicolor} \PY{o}{=} \PY{n}{dataset}\PY{p}{[}\PY{n}{dataset}\PY{p}{[}\PY{l+s+s2}{\PYZdq{}}\PY{l+s+s2}{species}\PY{l+s+s2}{\PYZdq{}}\PY{p}{]} \PY{o}{==} \PY{l+s+s2}{\PYZdq{}}\PY{l+s+s2}{Iris\PYZhy{}versicolor}\PY{l+s+s2}{\PYZdq{}}\PY{p}{]}
\PY{n}{versicolor}\PY{o}{.}\PY{n}{describe}\PY{p}{(}\PY{p}{)}
\end{Verbatim}
\end{tcolorbox}

            \begin{tcolorbox}[breakable, size=fbox, boxrule=.5pt, pad at break*=1mm, opacityfill=0]
\prompt{Out}{outcolor}{4}{\boxspacing}
\begin{Verbatim}[commandchars=\\\{\}]
       sepal\_length  sepal\_width  petal\_length  petal\_width
count     50.000000    50.000000     50.000000    50.000000
mean       5.936000     2.770000      4.260000     1.326000
std        0.516171     0.313798      0.469911     0.197753
min        4.900000     2.000000      3.000000     1.000000
25\%        5.600000     2.525000      4.000000     1.200000
50\%        5.900000     2.800000      4.350000     1.300000
75\%        6.300000     3.000000      4.600000     1.500000
max        7.000000     3.400000      5.100000     1.800000
\end{Verbatim}
\end{tcolorbox}
        
    \begin{tcolorbox}[breakable, size=fbox, boxrule=1pt, pad at break*=1mm,colback=cellbackground, colframe=cellborder]
\prompt{In}{incolor}{5}{\boxspacing}
\begin{Verbatim}[commandchars=\\\{\}]
\PY{c+c1}{\PYZsh{} extract statistics for virginica}
\PY{n}{virginica} \PY{o}{=} \PY{n}{dataset}\PY{p}{[}\PY{n}{dataset}\PY{p}{[}\PY{l+s+s2}{\PYZdq{}}\PY{l+s+s2}{species}\PY{l+s+s2}{\PYZdq{}}\PY{p}{]} \PY{o}{==} \PY{l+s+s2}{\PYZdq{}}\PY{l+s+s2}{Iris\PYZhy{}virginica}\PY{l+s+s2}{\PYZdq{}}\PY{p}{]}
\PY{n}{virginica}\PY{o}{.}\PY{n}{describe}\PY{p}{(}\PY{p}{)}
\end{Verbatim}
\end{tcolorbox}

            \begin{tcolorbox}[breakable, size=fbox, boxrule=.5pt, pad at break*=1mm, opacityfill=0]
\prompt{Out}{outcolor}{5}{\boxspacing}
\begin{Verbatim}[commandchars=\\\{\}]
       sepal\_length  sepal\_width  petal\_length  petal\_width
count      50.00000    50.000000     50.000000     50.00000
mean        6.58800     2.974000      5.552000      2.02600
std         0.63588     0.322497      0.551895      0.27465
min         4.90000     2.200000      4.500000      1.40000
25\%         6.22500     2.800000      5.100000      1.80000
50\%         6.50000     3.000000      5.550000      2.00000
75\%         6.90000     3.175000      5.875000      2.30000
max         7.90000     3.800000      6.900000      2.50000
\end{Verbatim}
\end{tcolorbox}
        
    \hypertarget{cvisualisations}{%
\subsubsection{c)Visualisations}\label{cvisualisations}}

\hypertarget{ipairplots}{%
\paragraph{i)Pairplots}\label{ipairplots}}

    \begin{tcolorbox}[breakable, size=fbox, boxrule=1pt, pad at break*=1mm,colback=cellbackground, colframe=cellborder]
\prompt{In}{incolor}{6}{\boxspacing}
\begin{Verbatim}[commandchars=\\\{\}]
\PY{c+c1}{\PYZsh{} import seaborn visualisation library}
\PY{k+kn}{import} \PY{n+nn}{seaborn} \PY{k}{as} \PY{n+nn}{sns}
\PY{k+kn}{import} \PY{n+nn}{matplotlib}\PY{n+nn}{.}\PY{n+nn}{pyplot} \PY{k}{as} \PY{n+nn}{plt}

\PY{c+c1}{\PYZsh{} generate scattergraph pairplot from data}
\PY{n}{plt}\PY{o}{.}\PY{n}{show}\PY{p}{(}
    \PY{n}{sns}\PY{o}{.}\PY{n}{pairplot}\PY{p}{(}\PY{n}{dataset}\PY{p}{,} \PY{n}{hue}\PY{o}{=}\PY{l+s+s2}{\PYZdq{}}\PY{l+s+s2}{species}\PY{l+s+s2}{\PYZdq{}}\PY{p}{,} \PY{n}{kind}\PY{o}{=}\PY{l+s+s2}{\PYZdq{}}\PY{l+s+s2}{scatter}\PY{l+s+s2}{\PYZdq{}}\PY{p}{)}
\PY{p}{)}

\PY{c+c1}{\PYZsh{} generate Kernel Density pairplot from data}
\PY{n}{plt}\PY{o}{.}\PY{n}{show}\PY{p}{(}
    \PY{n}{sns}\PY{o}{.}\PY{n}{pairplot}\PY{p}{(}\PY{n}{dataset}\PY{p}{,} \PY{n}{hue}\PY{o}{=}\PY{l+s+s2}{\PYZdq{}}\PY{l+s+s2}{species}\PY{l+s+s2}{\PYZdq{}}\PY{p}{,} \PY{n}{kind}\PY{o}{=}\PY{l+s+s2}{\PYZdq{}}\PY{l+s+s2}{kde}\PY{l+s+s2}{\PYZdq{}}\PY{p}{)}
\PY{p}{)}
\end{Verbatim}
\end{tcolorbox}

    \begin{center}
    \adjustimage{max size={0.9\linewidth}{0.9\paperheight}}{output_9_0.png}
    \end{center}
    { \hspace*{\fill} \\}
    
    \begin{center}
    \adjustimage{max size={0.9\linewidth}{0.9\paperheight}}{output_9_1.png}
    \end{center}
    { \hspace*{\fill} \\}
    
    These pairplots suggest two things: * Recorded values for sepal width
overlap significantly between all three species. As such, sepal width
may be considered a less significant attribute in terms of
identification. * There is a strong, nearly linear relationship between
petal length and petal width. This linearity suggests that this dataset
might be a good candidate for dimensionality reduction, e.g.~either
removing one of these attributes, or else by using a process like
Principal Component Analysis

    \hypertarget{iia-scatterplot-of-petal-length-x.-petal-width}{%
\paragraph{ii)A scatterplot of petal length x. petal
width}\label{iia-scatterplot-of-petal-length-x.-petal-width}}

    \begin{tcolorbox}[breakable, size=fbox, boxrule=1pt, pad at break*=1mm,colback=cellbackground, colframe=cellborder]
\prompt{In}{incolor}{7}{\boxspacing}
\begin{Verbatim}[commandchars=\\\{\}]
\PY{n}{plt}\PY{o}{.}\PY{n}{show}\PY{p}{(}
    \PY{n}{sns}\PY{o}{.}\PY{n}{scatterplot}\PY{p}{(}\PY{n}{data}\PY{o}{=}\PY{n}{dataset}\PY{p}{,} \PY{n}{x}\PY{o}{=}\PY{l+s+s2}{\PYZdq{}}\PY{l+s+s2}{petal\PYZus{}width}\PY{l+s+s2}{\PYZdq{}}\PY{p}{,} \PY{n}{y}\PY{o}{=}\PY{l+s+s2}{\PYZdq{}}\PY{l+s+s2}{petal\PYZus{}length}\PY{l+s+s2}{\PYZdq{}}\PY{p}{,} \PY{n}{hue}\PY{o}{=}\PY{l+s+s2}{\PYZdq{}}\PY{l+s+s2}{species}\PY{l+s+s2}{\PYZdq{}}\PY{p}{)}
\PY{p}{)}
\end{Verbatim}
\end{tcolorbox}

    \begin{center}
    \adjustimage{max size={0.9\linewidth}{0.9\paperheight}}{output_12_0.png}
    \end{center}
    { \hspace*{\fill} \\}
    
    \hypertarget{iii-linear-regression-of-petal-length-x-petal-width}{%
\paragraph{iii) Linear regression of petal length x petal
width}\label{iii-linear-regression-of-petal-length-x-petal-width}}

    \begin{tcolorbox}[breakable, size=fbox, boxrule=1pt, pad at break*=1mm,colback=cellbackground, colframe=cellborder]
\prompt{In}{incolor}{8}{\boxspacing}
\begin{Verbatim}[commandchars=\\\{\}]
\PY{n}{plt}\PY{o}{.}\PY{n}{show}\PY{p}{(}
    \PY{n}{sns}\PY{o}{.}\PY{n}{lmplot}\PY{p}{(}\PY{n}{data}\PY{o}{=}\PY{n}{dataset}\PY{p}{,} \PY{n}{x}\PY{o}{=}\PY{l+s+s2}{\PYZdq{}}\PY{l+s+s2}{petal\PYZus{}width}\PY{l+s+s2}{\PYZdq{}}\PY{p}{,} \PY{n}{y}\PY{o}{=}\PY{l+s+s2}{\PYZdq{}}\PY{l+s+s2}{petal\PYZus{}length}\PY{l+s+s2}{\PYZdq{}}\PY{p}{,} \PY{p}{)}
\PY{p}{)}
\end{Verbatim}
\end{tcolorbox}

    \begin{center}
    \adjustimage{max size={0.9\linewidth}{0.9\paperheight}}{output_14_0.png}
    \end{center}
    { \hspace*{\fill} \\}
    
    \begin{tcolorbox}[breakable, size=fbox, boxrule=1pt, pad at break*=1mm,colback=cellbackground, colframe=cellborder]
\prompt{In}{incolor}{9}{\boxspacing}
\begin{Verbatim}[commandchars=\\\{\}]
\PY{n}{plt}\PY{o}{.}\PY{n}{show}\PY{p}{(}
    \PY{n}{sns}\PY{o}{.}\PY{n}{lmplot}\PY{p}{(}\PY{n}{data}\PY{o}{=}\PY{n}{dataset}\PY{p}{,} \PY{n}{x}\PY{o}{=}\PY{l+s+s2}{\PYZdq{}}\PY{l+s+s2}{petal\PYZus{}width}\PY{l+s+s2}{\PYZdq{}}\PY{p}{,} \PY{n}{y}\PY{o}{=}\PY{l+s+s2}{\PYZdq{}}\PY{l+s+s2}{petal\PYZus{}length}\PY{l+s+s2}{\PYZdq{}}\PY{p}{,} \PY{n}{hue}\PY{o}{=}\PY{l+s+s2}{\PYZdq{}}\PY{l+s+s2}{species}\PY{l+s+s2}{\PYZdq{}} \PY{p}{)}
\PY{p}{)}
\end{Verbatim}
\end{tcolorbox}

    \begin{center}
    \adjustimage{max size={0.9\linewidth}{0.9\paperheight}}{output_15_0.png}
    \end{center}
    { \hspace*{\fill} \\}
    
    \hypertarget{d-clustering}{%
\subsubsection{d) clustering}\label{d-clustering}}

\hypertarget{i-k-means-of-petal-length-vs.-sepal-length}{%
\paragraph{i) K-means of petal length vs.~sepal
length}\label{i-k-means-of-petal-length-vs.-sepal-length}}

    \begin{tcolorbox}[breakable, size=fbox, boxrule=1pt, pad at break*=1mm,colback=cellbackground, colframe=cellborder]
\prompt{In}{incolor}{10}{\boxspacing}
\begin{Verbatim}[commandchars=\\\{\}]
\PY{k+kn}{from} \PY{n+nn}{sklearn}\PY{n+nn}{.}\PY{n+nn}{cluster} \PY{k+kn}{import} \PY{n}{KMeans}
\PY{c+c1}{\PYZsh{} from sklearn.metrics import silhouette\PYZus{}score}
\PY{k+kn}{from} \PY{n+nn}{sklearn}\PY{n+nn}{.}\PY{n+nn}{preprocessing} \PY{k+kn}{import} \PY{n}{StandardScaler}

\PY{c+c1}{\PYZsh{} create subset of data}
\PY{n}{petals\PYZus{}and\PYZus{}sepal\PYZus{}length} \PY{o}{=} \PY{n}{dataset}\PY{o}{.}\PY{n}{filter}\PY{p}{(}\PY{n}{items}\PY{o}{=}\PY{p}{[}\PY{l+s+s2}{\PYZdq{}}\PY{l+s+s2}{petal\PYZus{}length}\PY{l+s+s2}{\PYZdq{}}\PY{p}{,}\PY{l+s+s2}{\PYZdq{}}\PY{l+s+s2}{sepal\PYZus{}length}\PY{l+s+s2}{\PYZdq{}}\PY{p}{]}\PY{p}{)}

\PY{c+c1}{\PYZsh{} standardize values}
\PY{n}{scaler} \PY{o}{=} \PY{n}{StandardScaler}\PY{p}{(}\PY{p}{)}
\PY{n}{petals\PYZus{}and\PYZus{}sepal\PYZus{}length} \PY{o}{=} \PY{n}{scaler}\PY{o}{.}\PY{n}{fit\PYZus{}transform}\PY{p}{(}\PY{n}{petals\PYZus{}and\PYZus{}sepal\PYZus{}length}\PY{p}{)}

\PY{c+c1}{\PYZsh{} initialise kmeans class}
\PY{n}{kmeans} \PY{o}{=} \PY{n}{KMeans}\PY{p}{(}
    \PY{n}{init}\PY{o}{=}\PY{l+s+s2}{\PYZdq{}}\PY{l+s+s2}{random}\PY{l+s+s2}{\PYZdq{}}\PY{p}{,}
    \PY{n}{n\PYZus{}clusters}\PY{o}{=}\PY{l+m+mi}{3}\PY{p}{,}
    \PY{n}{n\PYZus{}init}\PY{o}{=}\PY{l+m+mi}{50}\PY{p}{,}
    \PY{n}{max\PYZus{}iter}\PY{o}{=}\PY{l+m+mi}{300}\PY{p}{,}
    \PY{n}{random\PYZus{}state}\PY{o}{=}\PY{k+kc}{None}\PY{p}{)}

\PY{c+c1}{\PYZsh{} run clustering algorithm}
\PY{n}{kmeans}\PY{o}{.}\PY{n}{fit}\PY{p}{(}\PY{n}{petals\PYZus{}and\PYZus{}sepal\PYZus{}length}\PY{p}{)}
\PY{n}{cluster\PYZus{}labels} \PY{o}{=} \PY{n}{kmeans}\PY{o}{.}\PY{n}{labels\PYZus{}}

\PY{c+c1}{\PYZsh{} add k means labels back to original dataframe}
\PY{n}{dataset}\PY{p}{[}\PY{l+s+s2}{\PYZdq{}}\PY{l+s+s2}{cluster\PYZus{}labels}\PY{l+s+s2}{\PYZdq{}}\PY{p}{]} \PY{o}{=} \PY{n}{cluster\PYZus{}labels}

\PY{c+c1}{\PYZsh{} visualise cluster\PYZus{}labels against recorded species}
\PY{n}{plt}\PY{o}{.}\PY{n}{show}\PY{p}{(}
    \PY{n}{sns}\PY{o}{.}\PY{n}{scatterplot}\PY{p}{(}\PY{n}{data}\PY{o}{=}\PY{n}{dataset}\PY{p}{,} \PY{n}{x}\PY{o}{=}\PY{l+s+s2}{\PYZdq{}}\PY{l+s+s2}{sepal\PYZus{}length}\PY{l+s+s2}{\PYZdq{}}\PY{p}{,} \PY{n}{y}\PY{o}{=}\PY{l+s+s2}{\PYZdq{}}\PY{l+s+s2}{petal\PYZus{}length}\PY{l+s+s2}{\PYZdq{}}\PY{p}{,} \PY{n}{hue}\PY{o}{=}\PY{l+s+s2}{\PYZdq{}}\PY{l+s+s2}{species}\PY{l+s+s2}{\PYZdq{}}\PY{p}{,} \PY{n}{style} \PY{o}{=}\PY{l+s+s2}{\PYZdq{}}\PY{l+s+s2}{cluster\PYZus{}labels}\PY{l+s+s2}{\PYZdq{}}\PY{p}{)}
\PY{p}{)}
\end{Verbatim}
\end{tcolorbox}

    \begin{center}
    \adjustimage{max size={0.9\linewidth}{0.9\paperheight}}{output_17_0.png}
    \end{center}
    { \hspace*{\fill} \\}
    
    \begin{itemize}
\item
  this figure shows the limitations of using a k-means algorithm with
  these two attributes.
\item
  whilst it is easy to see that each species broadly aligns with one of
  the identified clusters (e.g, iris setosa with cluster\_label 0) the
  model is far from perfect.
\item
  The 27 misidentified samples equate to an error rate of 18\%.
\end{itemize}

\hypertarget{k-means-clustering-of-petal-data}{%
\paragraph{K-means clustering of petal
data}\label{k-means-clustering-of-petal-data}}

    \begin{tcolorbox}[breakable, size=fbox, boxrule=1pt, pad at break*=1mm,colback=cellbackground, colframe=cellborder]
\prompt{In}{incolor}{11}{\boxspacing}
\begin{Verbatim}[commandchars=\\\{\}]
\PY{c+c1}{\PYZsh{} create subset of data}
\PY{n}{petals} \PY{o}{=} \PY{n}{dataset}\PY{o}{.}\PY{n}{filter}\PY{p}{(}\PY{n}{items}\PY{o}{=}\PY{p}{[}\PY{l+s+s2}{\PYZdq{}}\PY{l+s+s2}{petal\PYZus{}length}\PY{l+s+s2}{\PYZdq{}}\PY{p}{,}\PY{l+s+s2}{\PYZdq{}}\PY{l+s+s2}{petal\PYZus{}width}\PY{l+s+s2}{\PYZdq{}}\PY{p}{]}\PY{p}{)}

\PY{c+c1}{\PYZsh{} standardize values}
\PY{n}{petals} \PY{o}{=} \PY{n}{scaler}\PY{o}{.}\PY{n}{fit\PYZus{}transform}\PY{p}{(}\PY{n}{petals}\PY{p}{)}

\PY{c+c1}{\PYZsh{} run clustering algorithm}
\PY{n}{kmeans}\PY{o}{.}\PY{n}{fit}\PY{p}{(}\PY{n}{petals}\PY{p}{)}
\PY{n}{cluster\PYZus{}labels} \PY{o}{=} \PY{n}{kmeans}\PY{o}{.}\PY{n}{labels\PYZus{}}

\PY{c+c1}{\PYZsh{} add k means labels back to original dataframe}
\PY{n}{dataset}\PY{p}{[}\PY{l+s+s2}{\PYZdq{}}\PY{l+s+s2}{cluster\PYZus{}labels}\PY{l+s+s2}{\PYZdq{}}\PY{p}{]} \PY{o}{=} \PY{n}{cluster\PYZus{}labels}

\PY{c+c1}{\PYZsh{} visualise cluster\PYZus{}labels against recorded species}
\PY{n}{plt}\PY{o}{.}\PY{n}{show}\PY{p}{(}
    \PY{n}{sns}\PY{o}{.}\PY{n}{scatterplot}\PY{p}{(}\PY{n}{data}\PY{o}{=}\PY{n}{dataset}\PY{p}{,} \PY{n}{x}\PY{o}{=}\PY{l+s+s2}{\PYZdq{}}\PY{l+s+s2}{petal\PYZus{}width}\PY{l+s+s2}{\PYZdq{}}\PY{p}{,} \PY{n}{y}\PY{o}{=}\PY{l+s+s2}{\PYZdq{}}\PY{l+s+s2}{petal\PYZus{}length}\PY{l+s+s2}{\PYZdq{}}\PY{p}{,} \PY{n}{hue}\PY{o}{=}\PY{l+s+s2}{\PYZdq{}}\PY{l+s+s2}{species}\PY{l+s+s2}{\PYZdq{}}\PY{p}{,} \PY{n}{style} \PY{o}{=}\PY{l+s+s2}{\PYZdq{}}\PY{l+s+s2}{cluster\PYZus{}labels}\PY{l+s+s2}{\PYZdq{}}\PY{p}{)}
\PY{p}{)}
\end{Verbatim}
\end{tcolorbox}

    \begin{center}
    \adjustimage{max size={0.9\linewidth}{0.9\paperheight}}{output_19_0.png}
    \end{center}
    { \hspace*{\fill} \\}
    
    \begin{itemize}
\tightlist
\item
  the algorithm is much more succesful on these two attributes. The
  error rate is only 6 in 150, ie 96\% sucess rate.
\end{itemize}

\hypertarget{e-conclusion}{%
\subsubsection{e) Conclusion}\label{e-conclusion}}

This exploratory anaysis of the dataset has suggested that: * a strong
correlation exists between petal size and petal width, though the
strength of this correlation is significantly weaker at the species
level. * sepal width is one of the weaker attributes by which to
identify these three iris species. * K-means clustering (using petal
length and width as its axes) can create clusters that largely align
with species categorisation.

\hypertarget{task-2}{%
\subsection{Task 2}\label{task-2}}

\hypertarget{a-a-flowchart-of-the-steps-for-creating-a-dendogram-based-on-single-linkage-clustering.}{%
\subsubsection{a) A flowchart of the steps for creating a dendogram
based on single linkage
clustering.}\label{a-a-flowchart-of-the-steps-for-creating-a-dendogram-based-on-single-linkage-clustering.}}

Many tutorials for single-linkage clustering involve a redrawing the
distance matrix of all clusters after every clustering operation.
However, as single linkage is based on single links (ie, the distance
between single plots in differing clusters), no new information is
generated by this redrawing. If record of each point's clustering is
stored against it then the original distance matrix already contains all
the information needed to determine the next cluster. For clarity of
code, I have skipped the redundant step of redrawing the distance matrix
in each epoch. Names of the functions I will use to accomplish these
steps are on the left of the diagram.

\begin{figure}
\centering
\includegraphics{attachment:single\%20link\%20clustering\%20algorithm.svg}
\caption{single\%20link\%20clustering\%20algorithm.svg}
\end{figure}

    \hypertarget{b-subsetting-the-data}{%
\subsubsection{b) Subsetting the data}\label{b-subsetting-the-data}}

    \begin{tcolorbox}[breakable, size=fbox, boxrule=1pt, pad at break*=1mm,colback=cellbackground, colframe=cellborder]
\prompt{In}{incolor}{12}{\boxspacing}
\begin{Verbatim}[commandchars=\\\{\}]
\PY{c+c1}{\PYZsh{} filter out irrelevant attributes}
\PY{n}{sepals} \PY{o}{=} \PY{n}{dataset}\PY{o}{.}\PY{n}{filter}\PY{p}{(}\PY{n}{items}\PY{o}{=}\PY{p}{[}\PY{l+s+s2}{\PYZdq{}}\PY{l+s+s2}{sepal\PYZus{}length}\PY{l+s+s2}{\PYZdq{}}\PY{p}{,}\PY{l+s+s2}{\PYZdq{}}\PY{l+s+s2}{sepal\PYZus{}width}\PY{l+s+s2}{\PYZdq{}}\PY{p}{]}\PY{p}{)}

\PY{c+c1}{\PYZsh{} reducing set to first 6 entries}
\PY{n}{sepals} \PY{o}{=} \PY{n}{sepals}\PY{o}{.}\PY{n}{iloc}\PY{p}{[}\PY{p}{:}\PY{l+m+mi}{6}\PY{p}{]} 
\end{Verbatim}
\end{tcolorbox}

    \hypertarget{c-code-for-carrying-out-single-linkage-clustering.}{%
\subsubsection{c) Code for carrying out single linkage
clustering.}\label{c-code-for-carrying-out-single-linkage-clustering.}}

    \begin{tcolorbox}[breakable, size=fbox, boxrule=1pt, pad at break*=1mm,colback=cellbackground, colframe=cellborder]
\prompt{In}{incolor}{13}{\boxspacing}
\begin{Verbatim}[commandchars=\\\{\}]
\PY{k+kn}{import} \PY{n+nn}{math}

\PY{k}{class} \PY{n+nc}{Clustering}\PY{p}{:}

    \PY{k}{def} \PY{n+nf}{euclidean\PYZus{}distance}\PY{p}{(}\PY{n+nb+bp}{self}\PY{p}{,} \PY{n}{ax}\PY{p}{,} \PY{n}{ay}\PY{p}{,} \PY{n}{bx}\PY{p}{,} \PY{n}{by}\PY{p}{)}\PY{p}{:}
        \PY{n}{ed} \PY{o}{=} \PY{n}{math}\PY{o}{.}\PY{n}{sqrt}\PY{p}{(}\PY{p}{(}\PY{n+nb}{abs}\PY{p}{(}\PY{n}{ax}\PY{o}{\PYZhy{}}\PY{n}{bx}\PY{p}{)} \PY{o}{+} \PY{n+nb}{abs}\PY{p}{(}\PY{n}{ay}\PY{o}{\PYZhy{}}\PY{n}{by}\PY{p}{)}\PY{p}{)}\PY{p}{)}
        \PY{k}{return} \PY{n}{ed}

    \PY{k}{def} \PY{n+nf}{get\PYZus{}coord}\PY{p}{(}\PY{n+nb+bp}{self}\PY{p}{,} \PY{n}{point}\PY{p}{)}\PY{p}{:}
        
        \PY{n}{x} \PY{o}{=} \PY{n+nb+bp}{self}\PY{o}{.}\PY{n}{points}\PY{o}{.}\PY{n}{at}\PY{p}{[}\PY{n}{point}\PY{p}{,} \PY{l+s+s2}{\PYZdq{}}\PY{l+s+s2}{sepal\PYZus{}width}\PY{l+s+s2}{\PYZdq{}}\PY{p}{]}
        \PY{n}{y} \PY{o}{=} \PY{n+nb+bp}{self}\PY{o}{.}\PY{n}{points}\PY{o}{.}\PY{n}{at}\PY{p}{[}\PY{n}{point}\PY{p}{,} \PY{l+s+s2}{\PYZdq{}}\PY{l+s+s2}{sepal\PYZus{}length}\PY{l+s+s2}{\PYZdq{}}\PY{p}{]}

        \PY{k}{return} \PY{n}{x}\PY{p}{,}\PY{n}{y}

    \PY{k}{def} \PY{n+nf}{get\PYZus{}distance\PYZus{}matrix}\PY{p}{(}\PY{n+nb+bp}{self}\PY{p}{)}\PY{p}{:}
        
        \PY{c+c1}{\PYZsh{} create empty 2d matrix}
        \PY{n}{rows}\PY{p}{,} \PY{n}{cols} \PY{o}{=} \PY{p}{(}\PY{l+m+mi}{6}\PY{p}{,} \PY{l+m+mi}{6}\PY{p}{)}
        \PY{n}{dm} \PY{o}{=} \PY{p}{[}\PY{p}{[}\PY{l+m+mi}{0} \PY{k}{for} \PY{n}{i} \PY{o+ow}{in} \PY{n+nb}{range}\PY{p}{(}\PY{n}{cols}\PY{p}{)}\PY{p}{]} \PY{k}{for} \PY{n}{j} \PY{o+ow}{in} \PY{n+nb}{range}\PY{p}{(}\PY{n}{rows}\PY{p}{)}\PY{p}{]}

        \PY{c+c1}{\PYZsh{} fill in values }
        \PY{k}{for} \PY{n}{i} \PY{o+ow}{in} \PY{n+nb}{range}\PY{p}{(}\PY{n}{rows}\PY{p}{)}\PY{p}{:}
            \PY{n}{point\PYZus{}a} \PY{o}{=} \PY{n+nb+bp}{self}\PY{o}{.}\PY{n}{get\PYZus{}coord}\PY{p}{(}\PY{n}{i}\PY{p}{)}
            \PY{k}{for} \PY{n}{j} \PY{o+ow}{in} \PY{n+nb}{range}\PY{p}{(}\PY{n}{cols}\PY{p}{)}\PY{p}{:}

                \PY{c+c1}{\PYZsh{}this if/else is to stop doubling of information between identical i:j and j:i}
                \PY{k}{if} \PY{n}{i} \PY{o}{\PYZlt{}}\PY{o}{=} \PY{n}{j}\PY{p}{:}
                    \PY{n}{dm}\PY{p}{[}\PY{n}{i}\PY{p}{]}\PY{p}{[}\PY{n}{j}\PY{p}{]} \PY{o}{=} \PY{k+kc}{None}
                    \PY{k}{continue}
                \PY{k}{else}\PY{p}{:}
                    \PY{n}{point\PYZus{}b} \PY{o}{=} \PY{n+nb+bp}{self}\PY{o}{.}\PY{n}{get\PYZus{}coord}\PY{p}{(}\PY{n}{j}\PY{p}{)}
                    \PY{n}{dm}\PY{p}{[}\PY{n}{i}\PY{p}{]}\PY{p}{[}\PY{n}{j}\PY{p}{]} \PY{o}{=} \PY{n+nb+bp}{self}\PY{o}{.}\PY{n}{euclidean\PYZus{}distance}\PY{p}{(}\PY{o}{*}\PY{n}{point\PYZus{}a}\PY{p}{,} \PY{o}{*}\PY{n}{point\PYZus{}b}\PY{p}{)}
        
        \PY{c+c1}{\PYZsh{} remove blank fields}
        \PY{k}{for} \PY{n}{i} \PY{o+ow}{in} \PY{n}{dm}\PY{p}{:}
            \PY{k}{while} \PY{k+kc}{None} \PY{o+ow}{in} \PY{n}{i}\PY{p}{:}
                \PY{n}{i}\PY{o}{.}\PY{n}{remove}\PY{p}{(}\PY{k+kc}{None}\PY{p}{)}

        \PY{k}{return} \PY{n}{dm}
    
    \PY{k}{def} \PY{n+nf}{create\PYZus{}cluster\PYZus{}records}\PY{p}{(}\PY{n+nb+bp}{self}\PY{p}{)}\PY{p}{:}
        \PY{k}{for} \PY{n}{i} \PY{o+ow}{in} \PY{n+nb}{range}\PY{p}{(}\PY{n+nb}{len}\PY{p}{(}\PY{n+nb+bp}{self}\PY{o}{.}\PY{n}{points}\PY{p}{)}\PY{p}{)}\PY{p}{:}
            \PY{n+nb+bp}{self}\PY{o}{.}\PY{n}{points}\PY{o}{.}\PY{n}{at}\PY{p}{[}\PY{n}{i}\PY{p}{,} \PY{l+s+s2}{\PYZdq{}}\PY{l+s+s2}{cluster\PYZus{}record}\PY{l+s+s2}{\PYZdq{}}\PY{p}{]} \PY{o}{=} \PY{n}{i}
    
    \PY{k}{def} \PY{n+nf+fm}{\PYZus{}\PYZus{}init\PYZus{}\PYZus{}}\PY{p}{(}\PY{n+nb+bp}{self}\PY{p}{,} \PY{n}{dataframe}\PY{p}{)}\PY{p}{:}
        \PY{n+nb+bp}{self}\PY{o}{.}\PY{n}{points} \PY{o}{=} \PY{n}{dataframe}
        \PY{n+nb+bp}{self}\PY{o}{.}\PY{n}{distance\PYZus{}matrix} \PY{o}{=} \PY{n+nb+bp}{self}\PY{o}{.}\PY{n}{get\PYZus{}distance\PYZus{}matrix}\PY{p}{(}\PY{p}{)}
        \PY{n+nb+bp}{self}\PY{o}{.}\PY{n}{create\PYZus{}cluster\PYZus{}records}\PY{p}{(}\PY{p}{)}
        
    
    \PY{k}{def} \PY{n+nf}{next\PYZus{}cluster}\PY{p}{(}\PY{n+nb+bp}{self}\PY{p}{)}\PY{p}{:}
        
        \PY{c+c1}{\PYZsh{}abbreviated here for more legible code}
        \PY{n}{dm} \PY{o}{=} \PY{n+nb+bp}{self}\PY{o}{.}\PY{n}{distance\PYZus{}matrix}
        
        \PY{c+c1}{\PYZsh{} this tuple, once filled with appropriate values, will be passed back to caller}
        \PY{n}{next\PYZus{}cluster} \PY{o}{=} \PY{p}{\PYZob{}}
            \PY{l+s+s2}{\PYZdq{}}\PY{l+s+s2}{distance}\PY{l+s+s2}{\PYZdq{}} \PY{p}{:} \PY{l+m+mi}{99999}\PY{p}{,} \PY{c+c1}{\PYZsh{} an arbitrary high number }
            \PY{l+s+s2}{\PYZdq{}}\PY{l+s+s2}{i}\PY{l+s+s2}{\PYZdq{}} \PY{p}{:} \PY{k+kc}{None}\PY{p}{,}
            \PY{l+s+s2}{\PYZdq{}}\PY{l+s+s2}{j}\PY{l+s+s2}{\PYZdq{}} \PY{p}{:} \PY{k+kc}{None}\PY{p}{,}
        \PY{p}{\PYZcb{}}
        
        \PY{c+c1}{\PYZsh{} iterate over distance matrix}
        \PY{k}{for} \PY{n}{i} \PY{o+ow}{in} \PY{n+nb}{range}\PY{p}{(}\PY{n+nb}{len}\PY{p}{(}\PY{n}{dm}\PY{p}{)}\PY{p}{)}\PY{p}{:}
            \PY{k}{for} \PY{n}{j} \PY{o+ow}{in} \PY{n+nb}{range}\PY{p}{(}\PY{n+nb}{len}\PY{p}{(}\PY{n}{dm}\PY{p}{[}\PY{n}{i}\PY{p}{]}\PY{p}{)}\PY{p}{)}\PY{p}{:}
                
                \PY{c+c1}{\PYZsh{} \PYZsq{}is this the shortest distance we\PYZsq{}ve seen?}
                \PY{k}{if} \PY{n}{dm}\PY{p}{[}\PY{n}{i}\PY{p}{]}\PY{p}{[}\PY{n}{j}\PY{p}{]} \PY{o}{\PYZgt{}} \PY{n}{next\PYZus{}cluster}\PY{p}{[}\PY{l+s+s2}{\PYZdq{}}\PY{l+s+s2}{distance}\PY{l+s+s2}{\PYZdq{}}\PY{p}{]}\PY{p}{:}
                    \PY{k}{continue}
                
                \PY{c+c1}{\PYZsh{} \PYZsq{}are these two points in different clusters? \PYZsq{}}
                \PY{k}{elif} \PY{n+nb+bp}{self}\PY{o}{.}\PY{n}{points}\PY{o}{.}\PY{n}{at}\PY{p}{[}\PY{n}{i}\PY{p}{,} \PY{l+s+s2}{\PYZdq{}}\PY{l+s+s2}{cluster\PYZus{}record}\PY{l+s+s2}{\PYZdq{}}\PY{p}{]} \PY{o}{==} \PY{n+nb+bp}{self}\PY{o}{.}\PY{n}{points}\PY{o}{.}\PY{n}{at}\PY{p}{[}\PY{n}{j}\PY{p}{,} \PY{l+s+s2}{\PYZdq{}}\PY{l+s+s2}{cluster\PYZus{}record}\PY{l+s+s2}{\PYZdq{}}\PY{p}{]}\PY{p}{:}
                    \PY{k}{continue}
                
                \PY{c+c1}{\PYZsh{} record of best candidate for clustering based on this pass of dm so far}
                \PY{k}{else}\PY{p}{:}
                    \PY{n}{next\PYZus{}cluster} \PY{o}{=} \PY{p}{\PYZob{}}
                        \PY{l+s+s2}{\PYZdq{}}\PY{l+s+s2}{distance}\PY{l+s+s2}{\PYZdq{}} \PY{p}{:} \PY{n}{dm}\PY{p}{[}\PY{n}{i}\PY{p}{]}\PY{p}{[}\PY{n}{j}\PY{p}{]}\PY{p}{,}
                        \PY{l+s+s2}{\PYZdq{}}\PY{l+s+s2}{i}\PY{l+s+s2}{\PYZdq{}} \PY{p}{:} \PY{n}{i}\PY{p}{,}
                        \PY{l+s+s2}{\PYZdq{}}\PY{l+s+s2}{j}\PY{l+s+s2}{\PYZdq{}} \PY{p}{:} \PY{n}{j}\PY{p}{,}
                    \PY{p}{\PYZcb{}}
        \PY{k}{return} \PY{n}{next\PYZus{}cluster}
    
    \PY{k}{def} \PY{n+nf}{join\PYZus{}clusters}\PY{p}{(}\PY{n+nb+bp}{self}\PY{p}{,} \PY{n}{next\PYZus{}cluster}\PY{p}{)}\PY{p}{:}
        
        \PY{c+c1}{\PYZsh{}abbreviation for easier reading}
        \PY{n}{df} \PY{o}{=} \PY{n+nb+bp}{self}\PY{o}{.}\PY{n}{points}
        \PY{n}{i} \PY{o}{=} \PY{n}{next\PYZus{}cluster}\PY{p}{[}\PY{l+s+s2}{\PYZdq{}}\PY{l+s+s2}{i}\PY{l+s+s2}{\PYZdq{}}\PY{p}{]}
        \PY{n}{j} \PY{o}{=} \PY{n}{next\PYZus{}cluster}\PY{p}{[}\PY{l+s+s2}{\PYZdq{}}\PY{l+s+s2}{j}\PY{l+s+s2}{\PYZdq{}}\PY{p}{]}
        
        \PY{c+c1}{\PYZsh{} what clusters do points i and j belong to?}
        \PY{n}{i} \PY{o}{=} \PY{n}{df}\PY{o}{.}\PY{n}{at}\PY{p}{[}\PY{n}{i}\PY{p}{,} \PY{l+s+s2}{\PYZdq{}}\PY{l+s+s2}{cluster\PYZus{}record}\PY{l+s+s2}{\PYZdq{}}\PY{p}{]}
        \PY{n}{j} \PY{o}{=} \PY{n}{df}\PY{o}{.}\PY{n}{at}\PY{p}{[}\PY{n}{j}\PY{p}{,} \PY{l+s+s2}{\PYZdq{}}\PY{l+s+s2}{cluster\PYZus{}record}\PY{l+s+s2}{\PYZdq{}}\PY{p}{]}
        \PY{n+nb}{print}\PY{p}{(}\PY{n}{i}\PY{p}{,}\PY{n}{j}\PY{p}{)}
        
        \PY{c+c1}{\PYZsh{} join all of j\PYZsq{}s cluster to i\PYZsq{}s cluster}
        \PY{n+nb+bp}{self}\PY{o}{.}\PY{n}{points}\PY{o}{.}\PY{n}{replace}\PY{p}{(}\PY{p}{\PYZob{}}\PY{l+s+s2}{\PYZdq{}}\PY{l+s+s2}{cluster\PYZus{}record}\PY{l+s+s2}{\PYZdq{}}\PY{p}{:} \PY{n}{j}\PY{p}{,}\PY{p}{\PYZcb{}}\PY{p}{,} \PY{n}{i}\PY{p}{,} \PY{n}{inplace}\PY{o}{=}\PY{k+kc}{True}\PY{p}{)}
                
    \PY{k}{def} \PY{n+nf}{next\PYZus{}epoch}\PY{p}{(}\PY{n+nb+bp}{self}\PY{p}{)}\PY{p}{:}
        
        \PY{c+c1}{\PYZsh{} abbreviations}
        \PY{n}{df} \PY{o}{=} \PY{n+nb+bp}{self}\PY{o}{.}\PY{n}{points}
        
        \PY{c+c1}{\PYZsh{} find candidates for clustering }
        \PY{n}{next\PYZus{}cluster} \PY{o}{=} \PY{n+nb+bp}{self}\PY{o}{.}\PY{n}{next\PYZus{}cluster}\PY{p}{(}\PY{p}{)}
        \PY{n+nb}{print}\PY{p}{(}\PY{n}{next\PYZus{}cluster}\PY{p}{)}
        \PY{n+nb}{print}\PY{p}{(}\PY{l+s+s2}{\PYZdq{}}\PY{l+s+se}{\PYZbs{}n}\PY{l+s+s2}{\PYZdq{}}\PY{p}{)}
        
        \PY{c+c1}{\PYZsh{} make the cluster and print the cluster\PYZus{}record}
        \PY{n+nb+bp}{self}\PY{o}{.}\PY{n}{join\PYZus{}clusters}\PY{p}{(}\PY{n}{next\PYZus{}cluster}\PY{p}{)}
        \PY{n+nb}{print}\PY{p}{(}\PY{n}{df}\PY{o}{.}\PY{n}{sort\PYZus{}values}\PY{p}{(}\PY{l+s+s2}{\PYZdq{}}\PY{l+s+s2}{cluster\PYZus{}record}\PY{l+s+s2}{\PYZdq{}}\PY{p}{)}\PY{p}{)}
        \PY{n+nb}{print}\PY{p}{(}\PY{l+s+s2}{\PYZdq{}}\PY{l+s+se}{\PYZbs{}n}\PY{l+s+s2}{\PYZdq{}}\PY{p}{)}
        
        \PY{c+c1}{\PYZsh{} plot showing current clusters}
        \PY{n}{plt}\PY{o}{.}\PY{n}{show}\PY{p}{(}
            \PY{n}{sns}\PY{o}{.}\PY{n}{scatterplot}\PY{p}{(}\PY{n}{data}\PY{o}{=}\PY{n}{df}\PY{p}{,} \PY{n}{x}\PY{o}{=}\PY{l+s+s2}{\PYZdq{}}\PY{l+s+s2}{sepal\PYZus{}width}\PY{l+s+s2}{\PYZdq{}}\PY{p}{,} \PY{n}{y}\PY{o}{=}\PY{l+s+s2}{\PYZdq{}}\PY{l+s+s2}{sepal\PYZus{}length}\PY{l+s+s2}{\PYZdq{}}\PY{p}{,} \PY{n}{hue}\PY{o}{=}\PY{l+s+s2}{\PYZdq{}}\PY{l+s+s2}{cluster\PYZus{}record}\PY{l+s+s2}{\PYZdq{}}\PY{p}{,} \PY{n}{palette}\PY{o}{=}\PY{l+s+s2}{\PYZdq{}}\PY{l+s+s2}{Set2}\PY{l+s+s2}{\PYZdq{}}\PY{p}{)}
        \PY{p}{)}
        
        
        

                
                
\end{Verbatim}
\end{tcolorbox}

    \hypertarget{d-performing-initial-setup}{%
\subsubsection{d) Performing `Initial
Setup'}\label{d-performing-initial-setup}}

    \begin{tcolorbox}[breakable, size=fbox, boxrule=1pt, pad at break*=1mm,colback=cellbackground, colframe=cellborder]
\prompt{In}{incolor}{14}{\boxspacing}
\begin{Verbatim}[commandchars=\\\{\}]
\PY{c+c1}{\PYZsh{}initialise Clustering object using our data}
\PY{n}{c} \PY{o}{=} \PY{n}{Clustering}\PY{p}{(}\PY{n}{sepals}\PY{p}{)}

\PY{c+c1}{\PYZsh{}show cluster record just created}
\PY{n+nb}{print}\PY{p}{(}\PY{n}{c}\PY{o}{.}\PY{n}{points}\PY{p}{)}

\PY{c+c1}{\PYZsh{} plot of points coloured by cluster\PYZus{}record}
\PY{n}{plt}\PY{o}{.}\PY{n}{show}\PY{p}{(}
    \PY{n}{sns}\PY{o}{.}\PY{n}{scatterplot}\PY{p}{(}\PY{n}{data}\PY{o}{=}\PY{n}{c}\PY{o}{.}\PY{n}{points}\PY{p}{,} \PY{n}{x}\PY{o}{=}\PY{l+s+s2}{\PYZdq{}}\PY{l+s+s2}{sepal\PYZus{}width}\PY{l+s+s2}{\PYZdq{}}\PY{p}{,} \PY{n}{y}\PY{o}{=}\PY{l+s+s2}{\PYZdq{}}\PY{l+s+s2}{sepal\PYZus{}length}\PY{l+s+s2}{\PYZdq{}}\PY{p}{,} \PY{n}{hue}\PY{o}{=}\PY{l+s+s2}{\PYZdq{}}\PY{l+s+s2}{cluster\PYZus{}record}\PY{l+s+s2}{\PYZdq{}}\PY{p}{,} \PY{n}{palette}\PY{o}{=}\PY{l+s+s2}{\PYZdq{}}\PY{l+s+s2}{Set2}\PY{l+s+s2}{\PYZdq{}}\PY{p}{)}
        \PY{p}{)}
\end{Verbatim}
\end{tcolorbox}

    \begin{Verbatim}[commandchars=\\\{\}]
   sepal\_length  sepal\_width  cluster\_record
0           5.1          3.5             0.0
1           4.9          3.0             1.0
2           4.7          3.2             2.0
3           4.6          3.1             3.0
4           5.0          3.6             4.0
5           5.4          3.9             5.0
    \end{Verbatim}

    \begin{center}
    \adjustimage{max size={0.9\linewidth}{0.9\paperheight}}{output_26_1.png}
    \end{center}
    { \hspace*{\fill} \\}
    
    The cendogram looks like:
\includegraphics{attachment:dendrogram-Cluster\%200.svg}

    \begin{tcolorbox}[breakable, size=fbox, boxrule=1pt, pad at break*=1mm,colback=cellbackground, colframe=cellborder]
\prompt{In}{incolor}{15}{\boxspacing}
\begin{Verbatim}[commandchars=\\\{\}]
\PY{c+c1}{\PYZsh{}show Distance Matrix just generated}
\PY{k}{for} \PY{n}{i} \PY{o+ow}{in} \PY{n}{c}\PY{o}{.}\PY{n}{distance\PYZus{}matrix}\PY{p}{:}
    \PY{n+nb}{print}\PY{p}{(}\PY{n}{i}\PY{p}{)}
\end{Verbatim}
\end{tcolorbox}

    \begin{Verbatim}[commandchars=\\\{\}]
[]
[0.8366600265340751]
[0.8366600265340751, 0.6324555320336761]
[0.9486832980505138, 0.6324555320336765, 0.44721359549995865]
[0.44721359549995765, 0.8366600265340753, 0.8366600265340753, 0.948683298050514]
[0.8366600265340759, 1.1832159566199232, 1.1832159566199232, 1.264911064067352,
0.8366600265340757]
    \end{Verbatim}

    \hypertarget{e-clustering-epochs}{%
\subsubsection{e) Clustering Epochs}\label{e-clustering-epochs}}

\hypertarget{epoch-1}{%
\paragraph{Epoch 1}\label{epoch-1}}

    \begin{tcolorbox}[breakable, size=fbox, boxrule=1pt, pad at break*=1mm,colback=cellbackground, colframe=cellborder]
\prompt{In}{incolor}{16}{\boxspacing}
\begin{Verbatim}[commandchars=\\\{\}]
\PY{n}{c}\PY{o}{.}\PY{n}{next\PYZus{}epoch}\PY{p}{(}\PY{p}{)}
\end{Verbatim}
\end{tcolorbox}

    \begin{Verbatim}[commandchars=\\\{\}]
\{'distance': 0.44721359549995765, 'i': 4, 'j': 0\}


4.0 0.0
   sepal\_length  sepal\_width  cluster\_record
1           4.9          3.0             1.0
2           4.7          3.2             2.0
3           4.6          3.1             3.0
0           5.1          3.5             4.0
4           5.0          3.6             4.0
5           5.4          3.9             5.0


    \end{Verbatim}

    \begin{center}
    \adjustimage{max size={0.9\linewidth}{0.9\paperheight}}{output_30_1.png}
    \end{center}
    { \hspace*{\fill} \\}
    
    Dendogram looks like:
\includegraphics{attachment:dendrogram-Cluster\%201.svg} \#\#\#\# Epoch
2

    \begin{tcolorbox}[breakable, size=fbox, boxrule=1pt, pad at break*=1mm,colback=cellbackground, colframe=cellborder]
\prompt{In}{incolor}{17}{\boxspacing}
\begin{Verbatim}[commandchars=\\\{\}]
\PY{n}{c}\PY{o}{.}\PY{n}{next\PYZus{}epoch}\PY{p}{(}\PY{p}{)}
\end{Verbatim}
\end{tcolorbox}

    \begin{Verbatim}[commandchars=\\\{\}]
\{'distance': 0.44721359549995865, 'i': 3, 'j': 2\}


3.0 2.0
   sepal\_length  sepal\_width  cluster\_record
1           4.9          3.0             1.0
2           4.7          3.2             3.0
3           4.6          3.1             3.0
0           5.1          3.5             4.0
4           5.0          3.6             4.0
5           5.4          3.9             5.0


    \end{Verbatim}

    \begin{center}
    \adjustimage{max size={0.9\linewidth}{0.9\paperheight}}{output_32_1.png}
    \end{center}
    { \hspace*{\fill} \\}
    
    Dendogram looks like:
\includegraphics{attachment:dendrogram-Cluster\%202.svg} \#\#\#\#
Cluster 3

    \begin{tcolorbox}[breakable, size=fbox, boxrule=1pt, pad at break*=1mm,colback=cellbackground, colframe=cellborder]
\prompt{In}{incolor}{18}{\boxspacing}
\begin{Verbatim}[commandchars=\\\{\}]
\PY{n}{c}\PY{o}{.}\PY{n}{next\PYZus{}epoch}\PY{p}{(}\PY{p}{)}
\end{Verbatim}
\end{tcolorbox}

    \begin{Verbatim}[commandchars=\\\{\}]
\{'distance': 0.6324555320336761, 'i': 2, 'j': 1\}


3.0 1.0
   sepal\_length  sepal\_width  cluster\_record
1           4.9          3.0             3.0
2           4.7          3.2             3.0
3           4.6          3.1             3.0
0           5.1          3.5             4.0
4           5.0          3.6             4.0
5           5.4          3.9             5.0


    \end{Verbatim}

    \begin{center}
    \adjustimage{max size={0.9\linewidth}{0.9\paperheight}}{output_34_1.png}
    \end{center}
    { \hspace*{\fill} \\}
    
    Dendogram looks like:
\includegraphics{attachment:dendrogram-Cluster\%203.svg}

\hypertarget{epoch-4}{%
\paragraph{Epoch 4}\label{epoch-4}}

    \begin{tcolorbox}[breakable, size=fbox, boxrule=1pt, pad at break*=1mm,colback=cellbackground, colframe=cellborder]
\prompt{In}{incolor}{19}{\boxspacing}
\begin{Verbatim}[commandchars=\\\{\}]
\PY{n}{c}\PY{o}{.}\PY{n}{next\PYZus{}epoch}\PY{p}{(}\PY{p}{)}
\end{Verbatim}
\end{tcolorbox}

    \begin{Verbatim}[commandchars=\\\{\}]
\{'distance': 0.8366600265340751, 'i': 2, 'j': 0\}


3.0 4.0
   sepal\_length  sepal\_width  cluster\_record
0           5.1          3.5             3.0
1           4.9          3.0             3.0
2           4.7          3.2             3.0
3           4.6          3.1             3.0
4           5.0          3.6             3.0
5           5.4          3.9             5.0


    \end{Verbatim}

    \begin{center}
    \adjustimage{max size={0.9\linewidth}{0.9\paperheight}}{output_36_1.png}
    \end{center}
    { \hspace*{\fill} \\}
    
    Dendogram looks like:
\includegraphics{attachment:dendrogram-Cluster\%204-2.svg}

\hypertarget{epoch-5}{%
\paragraph{Epoch 5}\label{epoch-5}}

    \begin{tcolorbox}[breakable, size=fbox, boxrule=1pt, pad at break*=1mm,colback=cellbackground, colframe=cellborder]
\prompt{In}{incolor}{20}{\boxspacing}
\begin{Verbatim}[commandchars=\\\{\}]
\PY{n}{c}\PY{o}{.}\PY{n}{next\PYZus{}epoch}\PY{p}{(}\PY{p}{)}
\end{Verbatim}
\end{tcolorbox}

    \begin{Verbatim}[commandchars=\\\{\}]
\{'distance': 0.8366600265340757, 'i': 5, 'j': 4\}


5.0 3.0
   sepal\_length  sepal\_width  cluster\_record
0           5.1          3.5             5.0
1           4.9          3.0             5.0
2           4.7          3.2             5.0
3           4.6          3.1             5.0
4           5.0          3.6             5.0
5           5.4          3.9             5.0


    \end{Verbatim}

    \begin{center}
    \adjustimage{max size={0.9\linewidth}{0.9\paperheight}}{output_38_1.png}
    \end{center}
    { \hspace*{\fill} \\}
    
    Dendogram looks like:
\includegraphics{attachment:dendrogram-Complete.svg}

\hypertarget{check-of-my-finished-dendogram-against-another-library}{%
\paragraph{Check of my finished dendogram against another
library}\label{check-of-my-finished-dendogram-against-another-library}}

    \begin{tcolorbox}[breakable, size=fbox, boxrule=1pt, pad at break*=1mm,colback=cellbackground, colframe=cellborder]
\prompt{In}{incolor}{21}{\boxspacing}
\begin{Verbatim}[commandchars=\\\{\}]
\PY{k+kn}{from} \PY{n+nn}{scipy}\PY{n+nn}{.}\PY{n+nn}{cluster}\PY{n+nn}{.}\PY{n+nn}{hierarchy} \PY{k+kn}{import} \PY{n}{dendrogram}\PY{p}{,} \PY{n}{linkage}
\PY{n}{l} \PY{o}{=} \PY{n}{linkage}\PY{p}{(}\PY{n}{sepals}\PY{p}{)}
\PY{n}{plt}\PY{o}{.}\PY{n}{show}\PY{p}{(}\PY{n}{dendrogram}\PY{p}{(}\PY{n}{l}\PY{p}{,} \PY{n}{color\PYZus{}threshold}\PY{o}{=}\PY{l+m+mi}{0}\PY{p}{)}\PY{p}{)}
\end{Verbatim}
\end{tcolorbox}

    \begin{center}
    \adjustimage{max size={0.9\linewidth}{0.9\paperheight}}{output_40_0.png}
    \end{center}
    { \hspace*{\fill} \\}
    
    \begin{itemize}
\tightlist
\item
  My clustering, and the attached dendogram, matches that produced by
  SciPy in all except the final clustering.
\item
  In my plot I joined cluster (1,2,3) to (0,4) before joining the new
  cluster (0,1,2,3,4) to (5).
\item
  SciPy joined (1,2,3) to (0,4) at the same time as joining (5).
\item
  This can be explained by differences in rounding and/or floating point
  errors:

  \begin{itemize}
  \tightlist
  \item
    The distance of p5:p4, as printed in Epoch 5: 0.8366600265340757
  \item
    The distance of p2:p0, as printed in Epoch 4: 0.8366600265340751
  \item
    The difference between these two numbers according to Python:
    5.551115123125783e-16 ie 0.0000000000000005
  \item
    The difference between these numbers according to Google: 0
  \end{itemize}
\item
  As this difference in calculation is so minute, and has no practical
  effect on our conclusion, this discrepancy has been ignored.
\end{itemize}

\hypertarget{conclusion}{%
\subsubsection{Conclusion}\label{conclusion}}

\begin{itemize}
\tightlist
\item
  If we mark our cut-off point through the longest vertical line on the
  dendogram (excluding single value clusters, ie point 5) we create the
  following figure:
  \includegraphics{attachment:dendrogram-Speciation.svg}
\item
  We can describe the contents of the figure as either 3 clusters, or as
  2 clusters and an outlier.
\item
  If we were to treat this small sample size as representative of all
  irises, we could infer either two or three species from our dendogram.
\item
  It is clear, however, that our very small sample size, misrepresents
  the range of possible values. In doing so, it over-emphasizes the
  significance of differences between our six samples.
\item
  If we load the sepals information for the entire iris dataset, a very
  different context emerges:
\end{itemize}

    \begin{tcolorbox}[breakable, size=fbox, boxrule=1pt, pad at break*=1mm,colback=cellbackground, colframe=cellborder]
\prompt{In}{incolor}{22}{\boxspacing}
\begin{Verbatim}[commandchars=\\\{\}]
\PY{k}{for} \PY{n}{i} \PY{o+ow}{in} \PY{n+nb}{range} \PY{p}{(}\PY{l+m+mi}{6}\PY{p}{)}\PY{p}{:}
    \PY{n}{dataset}\PY{o}{.}\PY{n}{at}\PY{p}{[}\PY{n}{i}\PY{p}{,} \PY{l+s+s2}{\PYZdq{}}\PY{l+s+s2}{species}\PY{l+s+s2}{\PYZdq{}}\PY{p}{]} \PY{o}{=} \PY{l+s+s2}{\PYZdq{}}\PY{l+s+s2}{small\PYZus{}sample}\PY{l+s+s2}{\PYZdq{}}

\PY{n}{plt}\PY{o}{.}\PY{n}{show}\PY{p}{(}
    \PY{n}{sns}\PY{o}{.}\PY{n}{scatterplot}\PY{p}{(}\PY{n}{data}\PY{o}{=}\PY{n}{dataset}\PY{p}{,} \PY{n}{x}\PY{o}{=}\PY{l+s+s2}{\PYZdq{}}\PY{l+s+s2}{sepal\PYZus{}length}\PY{l+s+s2}{\PYZdq{}}\PY{p}{,} \PY{n}{y}\PY{o}{=}\PY{l+s+s2}{\PYZdq{}}\PY{l+s+s2}{sepal\PYZus{}width}\PY{l+s+s2}{\PYZdq{}}\PY{p}{,} \PY{n}{hue}\PY{o}{=}\PY{l+s+s2}{\PYZdq{}}\PY{l+s+s2}{species}\PY{l+s+s2}{\PYZdq{}}\PY{p}{,} \PY{n}{style}\PY{o}{=}\PY{l+s+s2}{\PYZdq{}}\PY{l+s+s2}{species}\PY{l+s+s2}{\PYZdq{}}\PY{p}{)}
\PY{p}{)}
\end{Verbatim}
\end{tcolorbox}

    \begin{center}
    \adjustimage{max size={0.9\linewidth}{0.9\paperheight}}{output_42_0.png}
    \end{center}
    { \hspace*{\fill} \\}
    
    \begin{itemize}
\tightlist
\item
  All members of our small sample size fall within the range for
  Iris-Setosa.
\item
  Checking the original dataset, it can be seen that all members of the
  small sample size do belong to Iris-Setosa.
\item
  Only one species is shown on our dendogram, though its focus on such a
  small range might overstate the importance of distances within that
  species.
\end{itemize}


    % Add a bibliography block to the postdoc
    
    
    
\end{document}